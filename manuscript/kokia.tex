% Template for PLoS
% Version 3.0 December 2014
%
% To compile to pdf, run:
% latex plos.template
% bibtex plos.template
% latex plos.template
% latex plos.template
% dvipdf plos.template
%
% % % % % % % % % % % % % % % % % % % % % %
%
% -- IMPORTANT NOTE
%
% This template contains comments intended 
% to minimize problems and delays during our production 
% process. Please follow the template instructions
% whenever possible.
%
% % % % % % % % % % % % % % % % % % % % % % % 
%
% Once your paper is accepted for publication, 
% PLEASE REMOVE ALL TRACKED CHANGES in this file and leave only
% the final text of your manuscript.
%
% There are no restrictions on package use within the LaTeX files except that 
% no packages listed in the template may be deleted.
%
% Please do not include colors or graphics in the text.
%
% Please do not create a heading level below \subsection. For 3rd level headings, use \paragraph{}.
%
% % % % % % % % % % % % % % % % % % % % % % %
%
% -- FIGURES AND TABLES
%
% Please include tables/figure captions directly after the paragraph where they are first cited in the text.
%
% DO NOT INCLUDE GRAPHICS IN YOUR MANUSCRIPT
% - Figures should be uploaded separately from your manuscript file. 
% - Figures generated using LaTeX should be extracted and removed from the PDF before submission. 
% - Figures containing multiple panels/subfigures must be combined into one image file before submission.
% See http://www.plosone.org/static/figureGuidelines for PLOS figure guidelines.
%
% Tables should be cell-based and may not contain:
% - tabs/spacing/line breaks within cells to alter layout or alignment
% - vertically-merged cells (no tabular environments within tabular environments, do not use \multirow)
% - colors, shading, or graphic objects
% See http://www.plosone.org/static/figureGuidelines#tables for table guidelines.
%
% For tables that exceed the width of the text column, use the adjustwidth environment as illustrated in the example table in text below.
%
% % % % % % % % % % % % % % % % % % % % % % % %
%
% -- EQUATIONS, MATH SYMBOLS, SUBSCRIPTS, AND SUPERSCRIPTS
%
% IMPORTANT
% Below are a few tips to help format your equations and other special characters according to our specifications. For more tips to help reduce the possibility of formatting errors during conversion, please see our LaTeX guidelines at http://www.plosone.org/static/latexGuidelines
%
% Please be sure to include all portions of an equation in the math environment.
%
% Do not include text that is not math in the math environment. For example, CO2 will be CO\textsubscript{2}.
%
% Please add line breaks to long display equations when possible in order to fit size of the column. 
%
% For inline equations, please do not include punctuation (commas, etc) within the math environment unless this is part of the equation.
%
% % % % % % % % % % % % % % % % % % % % % % % % 
%
% Please contact latex@plos.org with any questions.
%
% % % % % % % % % % % % % % % % % % % % % % % %

\documentclass[10pt,letterpaper]{article}
% \usepackage[top=0.85in,left=2.75in,footskip=0.75in]{geometry}

% Modifications for comments
% \usepackage[paperwidth=275.9mm, paperheight=279.4mm]{geometry} %regular letter size is 215.9 wide by 279.44 long
% \setlength{\evensidemargin}{95mm}
\usepackage[dvipsnames,svgnames,x11names]{xcolor}
\usepackage[markup=underlined]{changes}

\definechangesauthor[color=BrickRed]{Corrinne}
\definechangesauthor[color=NavyBlue]{Justin}
%%% Alternative definition to have the remarks
%%% in the margins instead of footnotes
\usepackage{todonotes}
\setlength{\marginparwidth}{3cm}
\makeatletter
\setremarkmarkup{\todo[color=Changes@Color#1!20,size=\scriptsize]{#1: #2}}
\makeatother

%% Rather hacky definition of a plain remark/note
%% by riding on \added
\newcommand{\note}[2][]{\added[id=#1,remark={#2}]{}}

% Use adjustwidth environment to exceed column width (see example table in text)
\usepackage{changepage}

% Use Unicode characters when possible
\usepackage[utf8]{inputenc}

% textcomp package and marvosym package for additional characters
\usepackage{textcomp,marvosym}

% fixltx2e package for \textsubscript
\usepackage{fixltx2e}

% amsmath and amssymb packages, useful for mathematical formulas and symbols
\usepackage{amsmath,amssymb}

% cite package, to clean up citations in the main text. Do not remove.
\usepackage{cite}

% Use nameref to cite supporting information files (see Supporting Information section for more info)
\usepackage{nameref,hyperref}

% line numbers
\usepackage[left]{lineno}

% ligatures disabled
\usepackage{microtype}
\DisableLigatures[f]{encoding = *, family = * }

% rotating package for sideways tables
\usepackage{rotating}

% Remove comment for double spacing
%\usepackage{setspace} 
%\doublespacing

% Text layout
\raggedright
\setlength{\parindent}{0.5cm}
\textwidth 5.25in 
\textheight 8.75in

% Bold the 'Figure #' in the caption and separate it from the title/caption with a period
% Captions will be left justified
\usepackage[aboveskip=1pt,labelfont=bf,labelsep=period,justification=raggedright,singlelinecheck=off]{caption}

% Use the PLoS provided BiBTeX style
\bibliographystyle{plos2009} 

% Remove brackets from numbering in List of References
\makeatletter
\renewcommand{\@biblabel}[1]{\quad#1.}
\makeatother

% Leave date blank
\date{}

% Header and Footer with logo
% \usepackage{lastpage,fancyhdr,graphicx}
% \pagestyle{myheadings}
% \pagestyle{fancy}
% \fancyhf{}
% \lhead{\includegraphics[natwidth=1.3in,natheight=0.4in]{PLOSlogo.png}}
% \rfoot{\thepage/\pageref{LastPage}}
% \renewcommand{\footrule}{\hrule height 2pt \vspace{2mm}}
% \fancyheadoffset[L]{2.25in}
% \fancyfootoffset[L]{2.25in}
% \lfoot{\sf PLOS}

%% Include all macros below

%% END MACROS SECTION


\begin{document}
\vspace*{0.35in}

% Title must be 150 characters or less
\begin{flushleft}
{\Large
\textbf\newline{Phylogenomic analysis of an unusual biogeographic disjunction in the cotton tribe (Gossypieae)}
}
\newline
% Insert Author names, affiliations and corresponding author email.
\\
Corrinne E Grover\textsuperscript{1},
Mark A Arick II\textsuperscript{1},
Justin Conover\textsuperscript{1},
Adam Thrasher\textsuperscript{1},
Guanjing Hu\textsuperscript{1},
William S Sanders\textsuperscript{1},
Rubab Naqvi\textsuperscript{1},
Muhammad Farooq\textsuperscript{1},
Joann Mudge\textsuperscript{1},
Thiru Ramaraj\textsuperscript{1},
Joshua A Udall\textsuperscript{1},
Daniel G Peterson\textsuperscript{1},
Jodi Scheffler\textsuperscript{1},
Brian Scheffler\textsuperscript{1},
Jonathan F Wendel\textsuperscript{1}
\\
\bf{1} Affiliation Dept/Program/Center, Institution Name, City, State, Country
\\
\bf{2} Affiliation Dept/Program/Center, Institution Name, City, State, Country
\\
\bf{3} Affiliation Dept/Program/Center, Institution Name, City, State, Country
\\

% Insert additional author notes using the symbols described below. Insert symbol callouts after author names as necessary.
% 
% Remove or comment out the author notes below if they aren't used.
%
* E-mail: Corresponding author@institute.edu
\end{flushleft}
% Please keep the abstract below 300 words
\section*{Abstract}


% Please keep the Author Summary between 150 and 200 words
% Use first person. PLOS ONE authors please skip this step. 
% Author Summary not valid for PLOS ONE submissions.   
\section*{Author Summary}


\linenumbers

\section*{Introduction}

One of the intriguing phenomena that characterizes the cotton tribe, Gossypieae,
is the prevalence of long-distance, trans-oceanic dispersals. The most famous of
these occur within the cotton genus itself (Gossypium); however, multiple events
are found throughout the tribe\cite{Dejoode1992, Fryxell1979, Stephens1958,
  Stephens1966, Wendel1989, Wendel1992, Wendel1990, Wendel1990b, Wendel2003,
  Seelanan1997}. The sister genera Kokia and Gossypioides both represent a
minimum of one such oceanic dispersal followed by individual regional
speciation. Based on molecular divergence estimates derived from both
chloroplast and nuclear genes, these genera collectively diverged from the
cotton genus during the Miocene approximately 10-15 million years ago (mya;
\cite{Seelanan1997, Cronn2002}), subsequently splitting into individual genera
and achieving widely dispersed, yet very localized ranges.

Kokia (Malvaceae) is a small Hawaiian endemic genus composed of four species
that were once widespread, major components of Hawaiian forests, yet are now all
either endangered, or recently extinct (K. lanceolata Lewton; \cite{Bates1990,
  Sherwood2014}). Few individuals remain of the two free-living extant species,
K. kauaiensis (Rock) Degener \& Duvel and K. drynarioides (Seem.) Lewton, the
latter of which is critically endangered and nearly extinct in the wild, while
the third endangered species, K. cookei Degener, exists only as a maintained
graft derived from a single individual (\cite{Service2012, Sherwood2014}). The
native region of its sister genus, Gossypioides, is located over 15,000
kilometers away in East Africa and Madagascar. The two species that comprise the
genus, G. kirkii M. Mast. and G. brevilanatum Hoch. (East Africa and Madagascar,
respectively), are themselves reproductively isolated and, with Kokia, are
cytologically distinct from the remainder of the cotton tribe in that they
appear to have experienced an aneuploid reduction in chromosome number.
Specifically, while most genera in the Gossypieae are based on n=13, species in
both Kokia and Gossypioides are n=12, likely representing a chromosome loss or
fusion event. The two species of Gossypioides also are cytogenetically distinct,
with an unusually long chromosome pair in G. brevilanatum \cite{Hutchinson1937,
  Hutchinson1943}.

Despite the extensive research on the evolution of Gossypium, these sister
genera have been grossly understudied, except in serving as phylogenetic
outgroups for cotton phylogenetic and genomic research \cite{Seelanan1997,
  Cronn2002}. Genomic resources in both genera are minimal, access to plant
material is limited, and with the recent exception of a study by Sherwood and
Morden (2014) on diversity among Kokia species, much of our knowledge regarding
these genera is decades old \cite{Hutchinson1947, Seelanan1997, Fryxell1968}.

The history of these genera, however, is intriguing. The current distribution of
Kokia in the Hawaiian Islands and Gossypioides in East Africa-Madagascar
necessitates at least one significant trans-oceanic traversal to a relatively
young island chain that began to emerge only about 3.4 mya, an age approximately
equivalent to the estimated divergence between Kokia drynarioides and
Gossypioides kirkii \cite{Seelanan1997} and slightly more recent than the basal
most divergence in Gossypium. Diversity within Gossypioides is unknown, aside
from acquisition of reproductive isolation between its sole two species;
however, diversity in Kokia has been evaluated for the purposes of conservation
\cite{Sherwood2014}. A remarkable amount of diversity within and among species
has been detected, particularly given the demographic history of Kokia, which
includes the original genetic bottleneck of the founder, range expansion, and
the subsequent bottleneck of habitat loss and the introduction of competitive
and/or damaging alien species \cite{Sherwood2014}.

Direct comparisons of these genera are limited. Hutchinson (1943) notes that
successful grafts can be made between Kokia drynarioides and Gossypioides
kirkii, and their shared chromosomal reduction (n=12) is unique in the tribe.
Estimates using a small number of nuclear genes suggest that genic distance
between K. drynarioides and G. kirkii are similar to estimates between basally
diverged species in Gossypium, i.e., approximately 2\% versus 3\%, although a
slight increase in replacement site substitutions is observed \cite{Cronn2002}.                                

Here we apply a whole-genome sequencing strategy to understanding the evolution
and divergence of these two genera, which collectively are the closest relatives
of the cotton genus  Gossypium. We present the first draft assembly of Kokia
drynarioides, and compare it to the sequence of Gossypioides kirkii (citation of
Gk paper). Through genome sequence comparisons, we derive a precise estimate of
the divergence between these two genera, and provide a foundation for a
reference sequence to use as a phylogenetic outgroup to Gossypium.



% You may title this section "Methods" or "Models". 
% "Models" is not a valid title for PLoS ONE authors. However, PLoS ONE
% authors may use "Analysis" 
\section*{Materials and Methods}
\subsection*{Kokia drynarioides sequencing and genome assembly}

DNA was extracted from mature leaves using the Qiagen Plant DNeasy kit (Qiagen).
Total genomic DNA was independently sheared via HOW into two average sizes,
i.e., 350bp and 550bp, for Illumina library construction. A single, independent
libraries was constructed from each fragment pool using the Illumina PCR-free
library construction kit (Illumina). The 350 bp library was sequenced on a
single lane of Illumina HiSeq2000 and the larger, 550bp library was sequenced on
two MiSeq flowcells (both at IGBB, Mississippi State University).

The data were trimmed and filtered with Trimmomatic v0.32 \cite{Bolger2014} with
the following options: (1) sequence adapter removal, (2) removal of leading
and/or trailing bases when the quality score (Q) \textless 28, (3) removal of
bases after average Q \textless 28 (8 nt window) or single base quality
\textless 10, and (4) removal of reads \textless 85 nt.

RNA was extracted \ldots. MEGAHIT commit:02102e1 \cite{Li2015} was used to assemble
the RNA data into transcripts.

The trimmed DNA data and RNA assembly were assembled via ABySS v2.0.1
\cite{Simpson2009}, using every 5th kmer value from 65 through 200. The assembly
with the highest E-size \cite{Salzberg2012} was retained for improvement and
analysis. Each retained assembly was further scaffolded with ABySS using the
MEGAHIT-derived transcripts. ABySS Sealer v2.0.1 \cite{Paulino2015} was used to
fill gaps in the scaffolded assembly using every 10th kmer starting at 100 and
decreasing to 30. Pilon v1.22 \cite{Walker2014} polished the resulting
gap-filled assembly using all trimmed DNA data. (Let's get this all into github)


\subsection*{Genome annotation}

MAKER (v2.31.6)\cite{Holt2011} annotation of the genome was completed in two
rounds, using only contigs of \textless 1 kb and training MAKER with
Kokia-specific sequences. First, pass de novo annotations were derived from
Genemark (v4.3.3)\cite{Lomsadze2005} and retained for MAKER training. At the
same time, BUSCO (v2)\cite{Simao2015} was used both to train Augustus and create
a Snap model\note[Corrinne]{WHAT'S A SNAP MODEL}. Finally, Trinity
\note[Corrinne]{WHY TRINITY VS MEGAHIT} (v2.2.0)\cite{Grabherr2011} was used to
create an RNASeq-assembly to pass to MAKER as EST evidence. The first pass of
MAKER was run using the combination of: (1) the output from Genemark, (2) the
BUSCO-generated Snap model, (3) the BUSCO-trained Augustus\cite{Stanke2003}
model, (4) the Trinity RNASeq-assembly as ESTs, and (5) the UniProt protein
database.

After the first pass of MAKER was complete, the annotations generated by MAKER
were passed to autoAug.pl, an annotation training script included with Augustus,
and were additionally used to generate a second Snap model. MAKER was run again
with the same input except using the newly generated Snap model (\#2 above) and
Augustus model (\#3 above) to replace those in the first pass. All annotations
were output to gff format and can be found at
https://github.com/Wendellab/KokiaKirkii.

\subsection*{Identification of Orthologs}

Amino acid sequences from G. kirkii, G. raimondii and K. drynarioides were
clustered using OrthoFinder v1.1.41 \cite{Emms2015}, which utilizes a Markov
clustering algorithm of normalized BLASTp scores to infer homology between
proteins sequences of different species. OrthoFinder is similar to OrthoMCL2
\cite{Li2003}, but reduces the number of BLAST results by filtering scores based
on reciprocal best hits (RBHs) and corrects for gene length biases and
floor-limitation of e-values in BLAST scores prior to clustering. These
corrections have been shown to increases precision by improved clustering of
singletons (i.e., groups in which only one gene from each species is present)
instead of entire gene families into a given orthologous group. Default values
were used for the inflation parameter (1.5) in the Markov clustering, and the
“–og” flag was used to prevent downstream analyses after the groups were
generated.


\subsection*{dN/dS Estimation and Timing of Divergence}

Singletons inferred from OrthoFinder were separated into all 3 possible pairwise
groups (Gr + Gk, Gr + Kd, Kd + Gk). Amino acid sequences from each pairwise
group were then aligned using the pairwise2 python package and the BLOSUM62
substitution matrix. The highest scoring alignments were then used as a guide to
codon-align the CDS sequences. The CODEML package in PAML \cite{Yang2007} was
used to calculate the dN, dS, and dN/dS values. Singletons in which any pairwise
comparison resulted in a dS value greater than 0.03\note[Justin]{May need to
  adjust after doing said analysis}\note[Corrinne]{What was our justification
  for this again?} was removed from the analysis and inferred to be a cluster of
non-orthologous proteins. Distributions of all pairwise dN, dS, and dN/dS values
were then plotted, and mean value and standard deviation is reported. Estimates
of total divergence time between each pairwise group was calculated using the
equation T=dS/(2r) where r is the absolute rate of synonymous substitutions of
Adh genes in palms (2.6 X 10-9 substitutions X substitution site-1 X year-1)
\cite{Cronn2002, Morton1996} or members of Brassicaceae (1.5 X 10-8
substitutions X synonymous site-1 X year-1) \cite{Koch2000}.


\subsection*{Copy Number Variation Estimation}
A custom Python script (https://github.com/Wendellab/KokiaKirkii) was used to
calculate lineage-specific gene losses and duplications as inferred by
OrthoFinder. A gene loss was defined as an orthologous group in which 2 species
had the same number of genes present (n), but the third species contained n-1
genes. Likewise, a gene duplication was identified by 2 species containing n
genes, while the third contained n+1.\note[Justin]{Very rough estimate of gene
  loss and duplication; do we want more sophisticated method? Other parts to
  this section?}\note[Corrinne]{We probably should cross-check these to make
  sure things didn’t get screwed up, e.g., a gene “loss” is actually where
  something got thrown in as a “duplication” or as a loner (true singleton with
  no match in other genomes)}


\subsection*{Repeat clustering and annotation}
All reads from one of the paired-end files (i.e., R1) were filtered for quality
and trimmed to a standard 95nt using Trimmomatic version 0.33 \cite{Bolger2014}
as per (https://github.com/Wendellab/KokiaKirkii). Surviving reads were randomly
subsampled to represent a 1\% genome size equivalent for each genome
\cite{Hendrix2005, Wendel2002} and combined as input into the RepeatExplorer
pipeline \cite{Novak2013, Novak2010}, which is designed to cluster reads based
on similarity and identify putative repetitive sequences using low-coverage,
small read sequencing. Clusters containing a minimum of 0.01\% of the total
input sequences (i.e., 201 reads from a total input of 2,013,469 reads) were
annotated by the RepeatExplorer implementation of RepeatMasker \cite{Smit2015}
using a custom library derived from a combination of Repbase version WHATEVER
\cite{Bao2015} and previously annotated cotton repeats \cite{Paterson2012,
  Grover2008, Grover2007, Grover2004, Hawkins2006}. A cutoff of 0.01\% read
representation is common; however, we evaluated the suitability of this cut
using a log of diminishing returns (FIGURE WHATEVER;
https://github.com/Wendellab/KokiaKirkii).

Within the annotated clusters, the number of megabases (Mb) attributable to that
cluster (i.e., element type) for each genome/accession was calculated based on
the 1\% genome representation of the sample and the standardized read length of
95 nt; total repetitive amounts for each broad repetitive classification were
summed from these results. The genome occupation of each cluster (i.e., the
calculated number of Mb) was normalized by genome size for each accession,
resulting in the percent of each genome occupied by that element type, for use
in multivariate visualization (i.e., Principle Coordinate Analysis and Principal
Component Analysis). All analyses were conducted in R \cite{R2017}; R versions
and scripts are available at (https://github.com/Wendellab/KokiaKirkii).

\subsection*{Repeat heterogeneity and relative age}
Relative cluster age was approximated using the among-read divergence profile of
each cluster, as previously used for Fritillaria \cite{Kelly2015} and dandelion
\cite{Ferreira2016}. Briefly, an all-versus-all BLASTn \cite{Boratyn2013,
  Altshul1990} was conducted on a cluster-by-cluster basis using the same BLAST
parameters implemented in RepeatExplorer. A histogram of pairwise percent
identity was generated for each cluster and the trend (i.e., biased toward
high-identity, “young” or lower-identity, “older” element reads) was described
for each via regression models using R. Specifically, two regression models were
used to describe the data as either linear ($Y = a + bX$) or quadratic
($Y = a + bX + cX^2$), and the model with the highest confidence was determined
via Bayesian Information Criterion \cite{Schwarz1978}. The read similarity
profile for each cluster was automatically evaluated for each histogram to
determine if the reads trend toward highly similar “young” or more divergent
“older” reads, as per (Julie paper) with an additional category. These
categories include (1) positive linear regression; (2) absence of linear
regression; (3) negative linear regression; (4) positive quadratic vertical
parabola, trend described by right-side of vertex; (4b) positive quadratic
vertical parabola, trend described by left-side of vertex; (5) negative
quadratic vertical parabola, trend described by right-side of vertex; and (6)
negative quadratic vertical parabola, trend described by left-side of vertex and
vertex at >99\% pairwise-identity (Figure WHATEVER). Categories which trend
toward highly identical reads (i.e., 1, 4, and 6) were interpreted as having
relatively young membership, whereas categories which trend toward lower
identity (i.e., 2, 3, 4b, and 5) were interpreted as being composed of older
elements. As with Ferreira de Carvalho (2016), this regression simply provides a
relative characterization of cluster/element age and is not designed to detect
statistically significant differences.

\subsection*{Repetitive profiles between Kokia drynarioides and Gossypioides kirkii}
Comparison of abundance for the annotated clusters in Kokia drynarioides and
Gossypioides kirkii were visualized via ggplot \cite{Wickham2016}, including a
1:1 ratio line to indicate the expected relationship between K. drynarioides and
G. kirkii cluster sizes if their repetitive profiles had remained static
post-divergence. Differential abundance (in read counts) between K. drynarioides
and G. kirkii for each cluster was evaluated via two-sample chi2 tests; all
p-values were subject to Benjamini-Hochberg correction for multiple testing
\cite{Benjamini2001}.


% Results and Discussion can be combined.
\section*{Results}

\subsection*{Kokia genome assembly and annotation}

\todo{STATS ON THE KOKIA GENOME HERE. STATS ON THE ANNOTATION TOO.}

\subsection*{Molecular evolution between Kokia drynarioides and Gossypioides
  kirkii}

\begin{enumerate}
\item Outgroup equivalency/utility: are they equal for molecular evolutionary purposes
  \begin{enumerate}
  \item Limited by no population data
  \item Ks/Ks of Gk-Gr versus Kd-Gr; are they equivalent
  \item Gene cluster comparisons: does Gk or Kd perform equivalently? i.e.,
    number of Gr-Kd only groups versus number of Gr-Gk only groups
  \item when would having two outgroups be of a benefit
  \item Ka/Ks for Gk-Kd: high or low? What do we expect?
  \item Gene content comparison : what is “missing”? What is unique?
  \end{enumerate}
\item Colinearity (at all?) or just intergenic SNPs/indels via gatk?
\end{enumerate}

\subsection*{Changes in the repetitive landscape between Kokia drynarioides and
  Gossypioides kirkii}

Because K. dryanarioides and G. kirkii have relatively compact genomes, multiple
representatives of three cotton species previously used for repetitive analysis
\cite{Renny-Byfield2016} were included in the clustering to aid in the
identification of repeat-derived sequences. Just over two million reads derived
from these five species (comprising 1\% genome size equivalents each) were
co-clustered using the RepeatExplorer pipeline, producing a total 74,001
clusters (n >2 reads). Because the smallest clusters are neither informative nor
reliable indicators of repetitiveness, we chose to annotate only those clusters
composed of greater than 0.01\% of the total reads input (=201 reads), resulting
in 274 retained clusters. We evaluated the cumulative read sum as the cluster
number increases (clusters are numbered from largest to smallest) to confirm
that this represents a reasonable partitioning of the data set.
\note{cotton\_cutoff.png}

Despite identically sized genomes, K. drynarioides and G. kirkii show an
approximately 1 Mb\note[Corrinne]{put the linear regression stuff in here?}
difference in clustered repeats, although this lacks statistical significance.
Contingency table analysis of the repetitive profiles of each species, as well
as the total amount of repetitive DNA calculated for each, suggest that these
profiles are indistinguishable (at p < 0.05), despite the intergeneric
comparison. Interspecies (intragenus) repetitive profiles for those Gossypium
species present in the analysis showed a different pattern, whereby the basally
divergent G. raimondii compared to either A-genome species (i.e., G. herbaceum
and G. arboreum) shows a highly distinct repetitive profile (p <0.05), although,
notably, the sister A-genome species are not distinct (see discussion).

To ascertain the extent of the differences between K. drynarioides and G.
kirkii, we considered the possibility that while the overall repetitive profiles
may not be significantly different, individual clusters may be. Toward this end,
we conducted a chi2 test of independence for each cluster and applied a
Benjamini-Hochberg correction for multiple testing. At p<0.05, XXX clusters (out
of XXX) are differentially abundant in K. drynarioides versus G. kirkii, with
the species displaying greater abundance occurring approximately the same number
of times for both (XXX with greater abundance in K. drynarioides versus XXX in
G. kirkii; Table Abundance). Because these differentially abundant clusters
could represent differences in either proliferation or decay/removal, we gauged
the relative age of each cluster based of the method of Ferreira de Carvalho
(2016). This analysis attempts to characterize the age of each
cluster\note[Corrinne]{should we redo this just for the Kok/Kirk reads? would
  the A-genome reads, minimally, be biasing some of these toward "youth"?} based
on the distinctiveness of the reads which comprise the cluster; that is, younger
clusters will have reads that are highly similar, whereas older clusters will
have reads that show a number of differences. While an imperfect measure, this
characterization permits a generalized perspective on the repeats identified
here. Overall, most of the repeats in K. drynarioides and G. kirkii displayed a
pattern suggestive of older elements (202 versus 72 “young”); however, of the
XXX differentially abundant clusters, XXX were categorized as “young” and XXX as
“older” (Table Abundance), potentially reflecting SOMETHING ABOUT GAIN VERSUS
LOSS.

Most of the clusters were broadly annotated as belonging to the Ty3/gypsy
superfamily, a result not surprising for a plant lineage (Figure Amounts).
Overall, gypsy elements comprise XXX to XXX of the K. drynarioides and G. kirkii
genomes, respectively, with uncategorized LTR-retrotransposons and Ty1/copia
elements comprising the next most abundant repeats and comprising similar
amounts in each genome. Unsurprisingly, the small genomes of K. drynarioides and
G. kirkii had lower absolute abundance of most repeat types except the predicted
non-LTR retroposons, in which these two species had comparable or slightly
greater occupation as the cotton species, which possess 2-3x larger genomes.
This difference is due to the sole retroposon clusters recovered, which was in
the top 5 largest clusters for both K. drynarioides and G. kirkii. The high
percent identity among reads for this cluster suggests it is relatively young,
and it has likely experienced proliferation in both species. Furthermore, the
cluster shows differential abundance between the two species, suggesting that
either the proliferation began prior to species divergence and continued with
varying success afterwards, or the two lineages experienced similar releases
from repression for this element, although again to varying degrees. The other
differentially abundant clusters were largely annotated as putative gypsy
elements (RIGHT?) (XX \%).

\section*{Discussion}

Divergence and speciation are expected outcomes of long-distance insular
dispersal, whose conceptual foundations are rooted in the observations of Darwin
and other early evolutionary biologists. The tribe Gossypieae is characterized
by such dispersals, ultimately achieving worldwide distribution on all tropical
and subtropical-inclusive continents. Most Gossypieae genera, save for the
eponymous Gossypium (cotton genus), have been grossly understudied except as
each pertains to the evolution of cotton. Here we present first-pass genome
assemblies for the outgroup congeners to Gossypium, which together provide
insight into the interesting biogeographic history of these genera and their
equivocality as outgroups in studying the evolution of the cotton genus.

\begin{enumerate}
\item Compare molecular differences to perceived degree of morphological differentiation?
\end{enumerate}

{
  \color{Blue}
Phylogenetics in the tribe: ndhF shows longer NJ branch length for Kokia than kirkii (congruence and consensus)

Long-distance salt water dispersal common in gossypieae


Advance Agronomy
\begin{itemize}
\item Lebronnecia – marquesas (south pacific)
\item Thespecia thespesioides – pan tropical
\item Hampia – neotropical (americas)
\item Thespesia populnea – pan tropical
\item Cephalohibiscus – new guinea and solomon islands (Australia)
\end{itemize}

Maybe we would expect there to be stepping speciation among these island
regions, e.g., south pacific lebronnecia to be between Kokia and kirkii, or
neotropical Hampea to be between the two. Clearly congeners, molecularly and
united by n=12. Hawaiian islands only $\approx$3myo, so Kokia probably colonized them as
they were formed. What about kirkii? Is it an older population, from which Kokia
is derived (probably not given the data), or was it a dispersal event from who
knows where of a now extinct ancestor?

}

  
\section*{Supporting Information}

% Include only the SI item label in the subsection heading. Use the \nameref{label} command to cite SI items in the text.
\subsection*{S1 Video}
\label{S1_Video}
{\bf Bold the first sentence.}  Maecenas convallis mauris sit amet sem ultrices gravida. Etiam eget sapien nibh. Sed ac ipsum eget enim egestas ullamcorper nec euismod ligula. Curabitur fringilla pulvinar lectus consectetur pellentesque.

\subsection*{S1 Text}
\label{S1_Text}
{\bf Lorem Ipsum.} Maecenas convallis mauris sit amet sem ultrices gravida. Etiam eget sapien nibh. Sed ac ipsum eget enim egestas ullamcorper nec euismod ligula. Curabitur fringilla pulvinar lectus consectetur pellentesque.

\subsection*{S1 Fig}
\label{S1_Fig}
{\bf Lorem Ipsum.} Maecenas convallis mauris sit amet sem ultrices gravida. Etiam eget sapien nibh. Sed ac ipsum eget enim egestas ullamcorper nec euismod ligula. Curabitur fringilla pulvinar lectus consectetur pellentesque.

\subsection*{S1 Table}
\label{S1_Table}
{\bf Lorem Ipsum.} Maecenas convallis mauris sit amet sem ultrices gravida. Etiam eget sapien nibh. Sed ac ipsum eget enim egestas ullamcorper nec euismod ligula. Curabitur fringilla pulvinar lectus consectetur pellentesque.

% Do NOT remove this, even if you are not including acknowledgments.
\section*{Acknowledgments}
Cras egestas velit mauris, eu mollis turpis pellentesque sit amet. Interdum et malesuada fames ac ante ipsum primis in faucibus. Nam id pretium nisi. Sed ac quam id nisi malesuada congue. Sed interdum aliquet augue, at pellentesque quam rhoncus vitae.

\nolinenumbers


\bibliography{kokia}{}

\end{document}

